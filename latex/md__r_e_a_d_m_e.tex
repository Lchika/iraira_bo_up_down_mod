\subsection*{概要}


\begin{DoxyItemize}
\item スレーブモジュールの共通部分だけ書いたコード
\end{DoxyItemize}

\subsection*{ピン関係}

\tabulinesep=1mm
\begin{longtabu}spread 0pt [c]{*{2}{|X[-1]}|}
\hline
\PBS\centering \cellcolor{\tableheadbgcolor}\textbf{ ピン番号  }&\PBS\centering \cellcolor{\tableheadbgcolor}\textbf{ 役割   }\\\cline{1-2}
\endfirsthead
\hline
\endfoot
\hline
\PBS\centering \cellcolor{\tableheadbgcolor}\textbf{ ピン番号  }&\PBS\centering \cellcolor{\tableheadbgcolor}\textbf{ 役割   }\\\cline{1-2}
\endhead
4  &モジュール通過判定   \\\cline{1-2}
5  &コース接触判定   \\\cline{1-2}
6  &D\+I\+Pスイッチbit0   \\\cline{1-2}
7  &D\+I\+Pスイッチbit1   \\\cline{1-2}
8  &D\+I\+Pスイッチbit2   \\\cline{1-2}
9  &D\+I\+Pスイッチbit3   \\\cline{1-2}
A4(18)  &D-\/sub2番ピン(S\+DA)   \\\cline{1-2}
A5(19)  &D-\/sub3番ピン(S\+CL)   \\\cline{1-2}
G\+ND  &D-\/sub1番ピン   \\\cline{1-2}
\end{longtabu}



\begin{DoxyItemize}
\item 上記ピン設定について\+D-\/sub関係以外は仮決め
\end{DoxyItemize}

\subsection*{D-\/sub関係}


\begin{DoxyItemize}
\item スレーブのアドレス指定:スタートモジュールに近い順に1,2,3...
\item D-\/sub関係は\+I2C(2,3番ピン)のみに集約し、\+D-\/subの4,5番ピン(ゴール通知・コース接触通知)は不要となった
\item ただし、上記対応により各モジュールの通過判定が必須になった
\begin{DoxyItemize}
\item どのモジュールと通信すればよいかわからないと\+I2\+C通信できないため
\end{DoxyItemize}
\end{DoxyItemize}

\subsection*{モジュール共通部分の処理について}


\begin{DoxyItemize}
\item D-\/sub関係の処理は{\ttfamily \mbox{\hyperlink{class_dsub_slave_communicator}{Dsub\+Slave\+Communicator}}}クラス内の処理で完結する
\item {\ttfamily \mbox{\hyperlink{iraira__bo__up__down_8ino_a7dfd9b79bc5a37d7df40207afbc5431f}{setup()}}}内で{\ttfamily \mbox{\hyperlink{class_dsub_slave_communicator}{Dsub\+Slave\+Communicator}}}クラスのインスタンス{\ttfamily $\ast$dsub\+Slave\+Communicator}を生成する
\item Dsub\+Slave\+Communicatorクラスは、マスタから通信開始通知を受け取ると、アクティブ状態になる
\item アクティブ状態でなければ、モジュールは何もする必要がない
\begin{DoxyItemize}
\item {\ttfamily \mbox{\hyperlink{iraira__bo__up__down_8ino_a0b33edabd7f1c4e4a0bf32c67269be2f}{loop()}}}内の以下のコードでアクティブ状態かどうか確認している \`{}\`{}\`{}c++ if(\+Dsub\+Slave\+Communicator\+::is\+\_\+active())\{ \`{}\`{}\`{}
\end{DoxyItemize}
\item D-\/sub関係処理は以下の部分で処理している 
\begin{DoxyCode}{0}
\DoxyCodeLine{\{c++\}}
\DoxyCodeLine{ dsubSlaveCommunicator->handle\_dsub\_event();}
\end{DoxyCode}

\begin{DoxyItemize}
\item コース接触判定、モジュール通過判定もここで行っている
\item この関数は定期的に呼ぶ必要がある
\end{DoxyItemize}
\end{DoxyItemize}

\subsection*{コンパイル時の注意事項}


\begin{DoxyItemize}
\item 以下の外部ライブラリが必要
\begin{DoxyItemize}
\item \href{https://www.arduinolibraries.info/libraries/arduino-stl}{\texttt{ Arduino\+S\+TL}}
\begin{DoxyItemize}
\item c++の標準ライブラリが使えるようになる
\begin{DoxyItemize}
\item c++の標準ライブラリ(S\+TL)については\href{https://ja.wikipedia.org/wiki/Standard_Template_Library}{\texttt{ Wikipedia}}参照
\end{DoxyItemize}
\item インストール方法
\begin{DoxyItemize}
\item \href{https://www.arduinolibraries.info/libraries/arduino-stl}{\texttt{ こちらのページ}}から{\ttfamily Arduino\+S\+T\+L-\/1.\+1.\+0.\+zip}をダウンロード
\item 解凍したフォルダを{\ttfamily C\+:\textbackslash{}Users\textbackslash{}your-\/username\textbackslash{}Documents\textbackslash{}Arduino\textbackslash{}libraries}(特に設定変えていなければここでよいはず)に置く
\end{DoxyItemize}
\end{DoxyItemize}
\end{DoxyItemize}
\end{DoxyItemize}

\subsection*{本プログラムのマニュアルについて}


\begin{DoxyItemize}
\item 本プログラムは\href{http://www.doxygen.nl/index.html}{\texttt{ Doxygen}}を使ってドキュメント化しています
\item マニュアルを見るにはプログラムダウンロード後、html/index.htmlをブラウザで開いてください 
\end{DoxyItemize}